\documentclass[letterpaper, 11pt]{article} % Document class with paper size and font size

% Preamble: Import necessary packages
\usepackage[utf8]{inputenc} % Input encoding
\usepackage[T1]{fontenc}    % Font encoding
\usepackage{amsmath, amssymb} % For math symbols and equations
\usepackage{geometry}       % For page layout
\geometry{margin=1.25in}    % Slightly larger margins (1.25 inches)
\usepackage{graphicx}       % For including graphics
\usepackage{hyperref}       % For hyperlinks
\usepackage{enumitem}       % For customizing lists
\usepackage{microtype}      % For better text alignment and kerning
\usepackage{lmodern}        % Use modern fonts for improved rendering

% Optional: Disable Synctex file creation
\synctex=0

% Title, Author, and Date
\title{\textbf{Fourier Transform Example}} % Bolded title for emphasis
\author{\textsc{Solomon Tessema}}         % Small caps for author name
\date{\today}

\begin{document}

% Title Page
\maketitle

\section*{Fourier Transform of a Sine Wave}

Given a sine wave in the time domain:
\[
x(t) = \sin(2 \pi f_0 t)
\]
where \( f_0 = 5 \, \text{Hz} \), we compute its Fourier Transform using the formula:
\[
X(f) = \int_{-\infty}^\infty x(t) e^{-j 2 \pi f t} \, dt
\]

\subsection*{Step 1: Rewrite the Sine Wave Using Euler's Formula}
Using Euler's formula:
\[
\sin(2 \pi f_0 t) = \frac{1}{2j} \left( e^{j 2 \pi f_0 t} - e^{-j 2 \pi f_0 t} \right)
\]
Substitute this into the Fourier Transform formula:
\[
X(f) = \int_{-\infty}^\infty \frac{1}{2j} \left( e^{j 2 \pi f_0 t} - e^{-j 2 \pi f_0 t} \right) e^{-j 2 \pi f t} \, dt
\]

\subsection*{Step 2: Simplify the Expression}
Distribute \( e^{-j 2 \pi f t} \):
\[
X(f) = \frac{1}{2j} \left( \int_{-\infty}^\infty e^{j 2 \pi (f_0 - f) t} \, dt - \int_{-\infty}^\infty e^{-j 2 \pi (f_0 + f) t} \, dt \right)
\]

\subsection*{Step 3: Solve Each Integral}
The Fourier Transform of a complex exponential is:
\[
\int_{-\infty}^\infty e^{j 2 \pi \alpha t} \, dt =
\begin{cases} 
    \infty, & \text{if } \alpha = 0, \\
    0, & \text{otherwise}.
\end{cases}
\]
1. For \( \int_{-\infty}^\infty e^{j 2 \pi (f_0 - f) t} \, dt \):
   \[
   \text{This evaluates to } \delta(f - f_0).
   \]
2. For \( \int_{-\infty}^\infty e^{-j 2 \pi (f_0 + f) t} \, dt \):
   \[
   \text{This evaluates to } \delta(f + f_0).
   \]

\subsection*{Step 4: Combine Results}
The Fourier Transform of \( x(t) = \sin(2 \pi f_0 t) \) is:
\[
X(f) = \frac{1}{2j} \left( \delta(f - f_0) - \delta(f + f_0) \right)
\]

\subsection*{Conclusion}
In the frequency domain, the sine wave has two peaks:
\begin{itemize}[label=\textbullet, leftmargin=*]
    \item A positive peak at \( f = f_0 \) (5 Hz).
    \item A negative peak at \( f = -f_0 \) (-5 Hz).
\end{itemize}

This representation shows that the signal oscillates at these two frequencies, and the Fourier Transform breaks down the time-domain signal into its frequency components.

\end{document}
